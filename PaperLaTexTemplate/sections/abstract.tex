\section*{基于LSM-Tree结构和Raft算法的分布式存储系统}
\section*{摘\ \ \ \ 要}

随着互联网的发展,网络用户的激增导致互联网服务提供公司需要存储极大规模的数据,而对于一些复杂场景下的数据复制以及分布式系统下数据库的可靠性仍然是一个巨大的挑战。
传统的互联网架构时代,单机数据库如MySQL,Oracle占领很大的市场份额,而单机数据库有很多缺点。
成本高:它们通常比分布式数据库系统更加复杂和成本更高,因为它们需要专门的硬件、存储设备和管理软件。
维护困难:由于它们是单独的系统,因此一旦出现问题,修复它们可能需要很长时间和高昂的成本。
不易扩展:当需要增加新功能或存储容量时,单机数据库系统可能无法轻松地扩展。
数据共享和备份困难:由于它们是单独的系统,数据不能轻松地在它们之间共享或备份。
安全性差:由于它们是单独的系统,数据很容易受到黑客攻击和数据泄露。
而分布式的数据存储恰如其分地解决了单机数据库的性能瓶颈问题和单机数据存储在实现面向用户系统时的各种痛点。


由于一个可靠的分布式存储系统设计复杂、挑战较大,本文致力于系统的存储策略,数据压缩方法,日志复制同步过程,Raft算法的可达性分析。
本文利用 Golang 编程语言的天然并发的特性, 来开发Raft 共识算法和 LSM-Tree 数据结构以实现分布式数据存储系统。 
本文的系统通过跨节点集群复制数据并使用 Raft 共识算法确保副本之间的一致性来提供容错性、可扩展性和高可用性。 
LSM-tree 数据结构通过优化磁盘访问和减少随机查找次数来实现高效的读写。 
通过精读论文原文,阅读参考Google开源C++版本的leveldb实现的LSM-Tree本文实现了分布式存储系统RAft Distributed Data Storage,简称为Radds。
本文的评估表明,本文的Radds系统在保持强一致性的同时实现了高性能和可扩展性,支持跨多个平台的客户端和多种语言的API。

\paragraph{关键词:} 日志结构归并树,Raft共识性算法,键值型数据存储,分布式系统

\clearpage


\section*{Distributed storage system based on LSM-Tree structure and Raft algorithm}

\section*{Abstract}

With the development of the Internet, the surge of network users has led to the need for Internet service providers to store extremely large-scale data. However, data replication in some complex scenarios and the reliability of databases in distributed systems are still a huge challenge.
In the era of traditional Internet architecture, stand-alone databases such as MySQL and Oracle occupy a large market share, but stand-alone databases have many shortcomings.
High cost: They are usually more complex and costly than distributed database systems because they require specialized hardware, storage devices, and management software.
Difficult to maintain: Since they are separate systems, it can take a long time and be costly to fix if something goes wrong.
Not easy to expand: When new functions or storage capacity need to be added, a stand-alone database system may not be easily expanded.
Difficulty in data sharing and backup: Since they are separate systems, data cannot be easily shared or backed up between them.
Poor security: Since they are separate systems, the data is vulnerable to hacking and data breaches.
Distributed data storage properly solves the performance bottleneck of stand-alone databases and various pain points of stand-alone data storage in implementing user-oriented systems.


Since the design of a reliable distributed storage system is complex and challenging, this paper focuses on the system's storage strategy, data compression method, log replication synchronization process, and the reachability analysis of the Raft algorithm.
This paper uses the natural concurrency of the Golang programming language to develop the Raft consensus algorithm and the LSM-Tree data structure to implement a distributed data storage system.
Our system provides fault tolerance, scalability, and high availability by replicating data across a cluster of nodes and using the Raft consensus algorithm to ensure consistency between replicas.
The LSM-tree data structure enables efficient reads and writes by optimizing disk access and reducing the number of random lookups.
Through intensive reading of the original text of the paper, read and refer to the LSM-Tree implemented by Google's open source C++ version of leveldb. This paper implements the distributed storage system RAft Distributed Data Storage, referred to as Radds.
Evaluations in this paper show that our Radds system achieves high performance and scalability while maintaining strong consistency, supporting clients across multiple platforms and APIs in multiple languages.

\paragraph{Keywords: }Log-Structured Merge-Tree,Raft consensus algorithm,Key-Value Data Storage,Distributed System







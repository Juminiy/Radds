    \subsection{共识层功能总体设计}	
    在一个分布式系统中,寻找可靠的共识性算法是分布式系统保证一致性(Consistency),可用性(Availability),分区容错性(Partition Tolerance)的关键性问题。
    对于共识性算法,本文选取容易理解和实现在真实系统之中的Raft算法,Reliable, Replicated, Redundant, Fault-Tolerant是raft算法的四个重要特性,也是算法的名字来源。
    
    Raft 节点总是处于三种状态之一:follower、candidate 或 leader。 
    所有节点最初都是作为跟随者开始的。 在这种状态下,节点可以接受来自领导者的日志条目并进行投票。 
    如果一段时间内没有收到任何条目,节点将自本文提升到候选状态。 在候选状态中,节点向其对等节点请求投票。 
    如果候选人获得法定人数的选票,则将其提升为领导者。 领导者必须接受新的日志条目并复制给所有其它追随者。 
    此外,如果过时的读取是不可接受的,则所有查询也必须在领导者上执行。

    一旦集群有了领导者,它就能够接受新的日志条目。 
    客户端可以请求领导者附加一个新的日志条目,这是一个不透明的二进制 blob 到 Raft。 
    领导者然后将条目写入持久存储并尝试复制到追随者的法定人数。 
    一旦日志条目被认为已提交,它就可以应用于有限状态机。 
    有限状态机是特定于应用程序的,并使用接口实现。
    
    一个明显的问题与复制日志的无限性质有关。 
    Raft 提供了一种机制,可以对当前状态进行快照,并压缩日志。 
    由于 FSM 抽象,恢复 FSM 的状态必须导致与重播旧日志相同的状态。 
    这允许 Raft 捕获某个时间点的 FSM 状态,然后删除所有用于达到该状态的日志。 
    这是自动执行的,无需用户干预,可防止无限制的磁盘使用,并最大限度地减少重放日志所花费的时间。
    
    最后,还有在新服务器加入或现有服务器离开时更新对等集的问题。 
    只要有法定数量的节点可用,这就不是问题,因为 Raft 提供了动态更新对等集合的机制。 
    如果节点法定人数不可用,那么这将成为一个非常具有挑战性的问题。 
    例如,假设只有 2 个对等点,A 和 B。仲裁大小也是 2,这意味着两个节点都必须同意提交日志条目。 
    如果 A 或 B 失败,则现在不可能达到法定人数。 
    这意味着集群无法添加或删除节点,或提交任何其它日志条目。 
    这导致不可用。 此时,需要手动干预以删除 A 或 B,并以引导模式重新启动其余节点。
    
    3 个节点的 Raft 集群可以容忍单个节点故障,而 5 个节点的集群可以容忍 2 个节点故障。 
    推荐的配置是运行 3 个或 5 个 raft 服务器。 这样可以在不显着牺牲性能的情况下最大限度地提高可用性。
    
    在性能方面,Raft 与 Paxos 不相上下。 
    假设领导稳定,提交日志条目需要单次往返集群的一半。 
    因此,性能受磁盘 I/O 和网络延迟的限制。

    % \subsection{客户端功能总体设计}

    %      本文的设计着重于服务端,对于客户端,只提供少部分轻量化的接口方便API调用和测试。

    %     \subsubsection{gRPC API客户端总体设计}
        
    %     gRPC API客户端是对于服务端进行接口抽象,以符合gRPC的消息格式来实现跨机器平台,跨操作系统,跨高级语言传输消息。
    %     \subsubsection{CLI 客户端总体设计}
    
    %     Command Line Interface 是为用户提供的一个命令行借口工具,支持集群加入,节点注册,数据Put/Delete/Get。
    % % \subsection{本章小结}

\clearpage
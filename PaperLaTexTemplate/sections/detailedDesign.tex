\section{Radds存储系统详细设计与实现}

	本章对存储系统的各层进行详细设计与实现,针对各层内部的子系统,各层之间的接口进行详细定义。

	\subsection{基础层详细设计与实现}
	
   		\subsubsection{错误处理的实现}

		针对go语言本身的特性,错误处理成为整个系统程序开发的首要项目,我们以轻量化,插件化的形式进行错误处理。
	
		\begin{lstlisting}[caption=Errors , label=code_radds_errors]

 
		\end{lstlisting}                 

	
			
   		\subsubsection{日志系统的实现}
    
	   	为了防止写入内存的数据库因为进程异常、操作系统掉电等情况发生丢失,
	   	存储系统在写内存之前会将本次写操作的内容写入日志文件中。
    
    	\begin{figure}[H]
    		\centering
    		\includegraphics[width=0.95\textwidth]{images/two_log}
    		\caption{日志系统架构图}
    		\label{two_log}
    	\end{figure}
		存储系统中,有两个memory db,以及对应的两份日志文件。
		其中一个memory db是可读写的,当这个db的数据量超过预定的上限时,
		便会转换成一个不可写的memory db,与此同时,与之对应的日志文件也变成一份frozen log。

		而新生成的immutable memory db则会由后台的minor compaction进程将其转换成一个sstable文件进行持久化,
		持久化完成,与之对应的frozen log被删除。
    
		\begin{enumerate}
		\item 日志结构

		\begin{figure}[H]
			\centering
			\includegraphics[width=0.95\textwidth]{images/journal}
			\caption{日志文件存储结构图}
			\label{journal}
		\end{figure}
		为了增加读取效率,日志文件中按照block进行划分,每个block的大小为32KiB。
		每个block中包含了若干个完整的chunk。
		
		一条日志记录包含一个或多个chunk。
		每个chunk包含了一个7字节大小的header,前4字节是该chunk的校验码,
		紧接的2字节是该chunk数据的长度,以及最后一个字节是该chunk的类型。
		其中checksum校验的范围包括chunk的类型以及随后的data数据。
	
		chunk共有四种类型:full,first,middle,last。
		一条日志记录若只包含一个chunk,则该chunk的类型为full。
		若一条日志记录包含多个chunk,则这些chunk的第一个类型为first, 
		最后一个类型为last,中间包含大于等于0个middle类型的chunk。
		
		由于一个block的大小为32KiB,因此当一条日志文件过大时,
		会将第一部分数据写在第一个block中,且类型为first,
		若剩余的数据仍然超过一个block的大小,则第二部分数据写在第二个block中,
		类型为middle,最后剩余的数据写在最后一个block中,类型为last。
		
		\item 日志内容
		
		日志的内容为写入的batch编码后的信息。

		具体的格式为:

		\begin{figure}[H]
			\centering
			\includegraphics[width=0.95\textwidth]{images/journal_content}
			\caption{日志文件格式图}
			\label{journal_content}
		\end{figure}

		一条日志记录的内容包含:Header和Data
		其中Header中有(1)当前db的sequence number(2)本次日志记录中所包含的put/del操作的个数。
		
		紧接着写入所有batch编码后的内容。
		\item 日志文件写 
		
		\begin{figure}[H]
			\centering
			\includegraphics[width=0.95\textwidth]{images/journal_write}
			\caption{日志文件写流程图}
			\label{journal_write}
		\end{figure}
		
		日志写入流程较为简单,在存储系统内部,实现了一个journal的writer。
		首先调用Next函数获取一个singleWriter,
		这个singleWriter的作用就是写入一条journal记录。

		singleWriter开始写入时,标志着第一个chunk开始写入。
		在写入的过程中,不断判断writer中buffer的大小,若超过32KiB,
		将chunk开始到现在做为一个完整的chunk,为其计算header之后将整个chunk写入文件。
		与此同时reset buffer,开始新的chunk的写入。

		若一条journal记录较大,则可能会分成几个chunk存储在若干个block中。

		\item 日志文件读 
		
		同样,日志读取也较为简单。为了避免频繁的IO读取,每次从文件中读取数据时,
		按block(32KiB)进行块读取。

		每次读取一条日志记录,reader调用Next函数返回一个singleReader。
		singleReader每次调用Read函数就返回一个chunk的数据。每次读取一个chunk,
		都会检查这批数据的校验码、数据类型、数据长度等信息是否正确,若不正确,
		且用户要求严格的正确性,则返回错误,否则丢弃整个chunk的数据。

		循环调用singleReader的read函数,直至读取到一个类型为Last的chunk,
		表示整条日志记录都读取完毕,返回。

		\begin{figure}[H]
			\centering
			\includegraphics[width=0.95\textwidth]{images/journal_read}
			\caption{日志文件读流程图}
			\label{journal_read}
		\end{figure}
		
		
	
		\end{enumerate}
	
   		\subsubsection{其他工具库的实现}
    


  	\subsection{存储层详细设计与实现}
	
		\subsubsection{写数据的实现}
		
		\begin{enumerate}
		\item 写入数据的整体流程
			
		先来分析一下存储系统整个写入的流程,底层数据结构的支持以及为何能够优化我们的写入性能。
		
		\begin{figure}[H]
			\centering
			\includegraphics[width=0.95\textwidth]{images/write_op}
			\caption{存储系统写数据流程}
			\label{write_op}
		\end{figure}

		数据的一次写入分为两部分:

		将写操作写入日志;
		将写操作应用到内存数据库中;
		之前已经阐述过为何这样的操作可以优化写入性能,以及通过先写日志的方法能够保障用户的写入不丢失。

		其实仍然存在写入丢失的隐患。在写设置为非同步的情况下,在写完日志文件以后,
		操作系统并不是直接将这些数据真正落到磁盘中,而是暂时留在操作系统缓存中,
		因此当用户写入操作完成,操作系统还未来得及落盘的情况下,发生系统宕机,就会造成写丢失;
		但是若只是进程异常退出,则不存在该问题。

		\item 写类型
		
		由于是键值型非关系型数据存储,存储系统对外提供的写入接口有:
		(1)Put(2)Delete两种。这两种本质对应同一种操作,
		Delete操作同样会被转换成一个value为空的Put操作。
	
		除此以外,我们还提供了一个批量处理的工具Batch,用户可以依据Batch来完成批量更新操作,
		且这些操作是原子性的。

		\item batch结构
		
		无论是Put/Del操作,还是批量操作,
		底层都会为这些操作创建一个batch实例作为一个数据库操作的最小执行单元。
		因此首先介绍一下batch的组织结构。

		\begin{figure}[H]
			\centering
			\includegraphics[width=0.95\textwidth]{images/batch}
			\caption{batch的数据结构}
			\label{batch}
		\end{figure}

		在batch中,每一条数据项都按照上图格式进行编码。
		每条数据项编码后的第一位是这条数据项的类型(更新还是删除),
		之后是数据项key的长度,数据项key的内容;若该数据项不是删除操作,
		则再加上value的长度,value的内容。

		batch中会维护一个size值,用于表示其中包含的数据量的大小。
		该size值为所有数据项key与value长度的累加,以及每条数据项额外的8个字节。
		这8个字节用于存储一条数据项额外的一些信息。
		

		\item key值编码
		
		当数据项从batch中写入到内存数据库中时,需要将一个key值的转换,即在存储系统内部,
		所有数据项的key是经过特殊编码的,这种格式称为internalKey。

		\begin{figure}[H]
			\centering
			\includegraphics[width=0.95\textwidth]{images/internalkey}
			\caption{internalkey的数据结构}
			\label{internalkey}
		\end{figure}


		internalkey在用户key的基础上,尾部追加了8个字节,
		用于存储(1)该操作对应的sequence number(2)该操作的类型。

		其中,每一个操作都会被赋予一个sequence number。
		该计时器是在存储系统内部维护,每进行一次操作就做一个累加。
		由于在存储系统中,一次更新或者一次删除,采用的是append的方式,
		并非直接更新原数据。因此对应同样一个key,会有多个版本的数据记录,
		而最大的sequence number对应的数据记录就是最新的。

		此外,存储系统的快照(snapshot)也是基于这个sequence number实现的,
		即每一个sequence number代表着数据库的一个版本。

		\item 数据合并写入
		
		存储系统中,在面对并发写入时,做了一个处理的优化。
		在同一个时刻,只允许一个写入操作将内容写入到日志文件以及内存数据库中。
		为了在写入进程较多的情况下,减少日志文件的小写入,增加整体的写入性能,
		存储系统将一些“小写入”合并成一个“大写入”。

		当前写操作

		(1)第一个写入操作获取到写入锁;
		
		(2)在当前写操作的数据量未超过合并上限,且有其他写操作pending的情况下,
		将其他写操作的内容合并到自身;
		
		(3)若本次写操作的数据量超过上限,或者无其他pending的写操作了,
		将所有内容统一写入日志文件,并写入到内存数据库中;
		
		(4)通知每一个被合并的写操作最终的写入结果,释放或移交写锁;

		其它写操作

		(1)等待获取写锁或者被合并;
		
		(2)若被合并,判断是否合并成功,若成功,则等待最终写入结果;
		
		(3)反之,则表明获取锁的写操作已经oversize了,
		此时,该操作直接从上个占有锁的写操作中接过写锁进行写入;
		
		(4)若未被合并,则继续等待写锁或者等待被合并;

		\begin{figure}[H]
			\centering
			\includegraphics[width=0.95\textwidth]{images/write_merge}
			\caption{写合并的流程}
			\label{write_merge}
		\end{figure}

		\item 原子性
		
		存储系统的任意一个写操作(无论包含了多少次写),其原子性都是由日志文件实现的。
		一个写操作中所有的内容会以一个日志中的一条记录,作为最小单位写入。

		考虑以下两种异常情况:

		(1)写日志未开始,或写日志完成一半,进程异常退出;
		(2)写日志完成,进程异常退出;

		前者中可能存储一个写操作的部分写已经被记载到日志文件中,仍然有部分写未被记录,
		这种情况下,当数据库重新启动恢复时,读到这条日志记录时,发现数据异常,直接丢弃或退出,
		实现了写入的原子性保障。

		后者,写日志已经完成,写入日志的数据未真正持久化,
		存储系统启动恢复时通过redo日志实现数据写入,仍然保障了原子性。

		\end{enumerate}


		\subsubsection{读数据的实现}

	存储系统提供给用户两种进行读取数据的接口:

	直接通过Get接口读取数据;
	首先创建一个snapshot,基于该snapshot调用Get接口读取数据;
	两者的本质是一样的,只不过第一种调用方式默认地以当前数据库的状态创建了一个snapshot,
	并基于此snapshot进行读取。

	读者可能不了解snapshot(快照)到底是什么?简单地来说,就是数据库在某一个时刻的状态。
	基于一个快照进行数据的读取,读到的内容不会因为后续数据的更改而改变。
	
	由于两种方式本质都是基于快照进行读取的,因此在介绍读操作之前,首先介绍快照。

		\begin{enumerate}
		\item snapshot(快照)
		
		快照代表着数据库某一个时刻的状态,在存储系统中,巧妙地用一个整型数来代表一个数据库状态。

		在存储系统中,用户对同一个key的若干次修改(包括删除)是以维护多条数据项的方式进行存储的(直至进行compaction时才会合并成同一条记录),
		每条数据项都会被赋予一个序列号,代表这条数据项的新旧状态。一条数据项的序列号越大,
		表示其中代表的内容为最新值。

		因此,每一个序列号,其实就代表着存储系统的一个状态。
		换句话说,每一个序列号都可以作为一个状态快照。

		当用户主动或者被动地创建一个快照时,存储系统会以当前最新的序列号对其赋值。
		例如图中用户在序列号为98的时刻创建了一个快照,
		并且基于该快照读取key为“name”的数据时,即便此刻用户将"name"的值修改为"dog",
		再删除,用户读取到的内容仍然是“cat”。

		\begin{figure}[H]
			\centering
			\includegraphics[width=0.95\textwidth]{images/snapshot}
			\caption{快照数据示例图}
			\label{snapshot}
		\end{figure}

		所以,利用快照能够保证数据库进行并发的读写操作。

		在获取到一个快照之后,存储系统会为本次查询的key构建一个internalKey(格式如上文所述),
		其中internalKey的seq字段使用的便是快照对应的seq。
		通过这种方式可以过滤掉所有seq大于快照号的数据项。
		

		\item 读数据流程
		
		\begin{figure}[H]
			\centering
			\includegraphics[width=0.95\textwidth]{images/readop}
			\caption{读数据流程}
			\label{readop}
		\end{figure}

		存储系统读取分为三步:

		(1)在memory db中查找指定的key,若搜索到符合条件的数据项,结束查找;

		(2)在冻结的memory db中查找指定的key,若搜索到符合条件的数据项,结束查找;
		
		(3)按低层至高层的顺序在level i层的sstable文件中查找指定的key,
		若搜索到符合条件的数据项,结束查找,否则返回Not Found错误,表示数据库中不存在指定的数据;

		注意存储系统在每一层sstable中查找数据时,都是按序依次查找sstable的。

		0层的文件比较特殊。由于0层的文件中可能存在key重合的情况,
		因此在0层中,文件编号大的sstable优先查找。
		理由是文件编号较大的sstable中存储的总是最新的数据。

		非0层文件,一层中所有文件之间的key不重合,
		因此存储系统可以借助sstable的元数据(一个文件中最小与最大的key值)进行快速定位,
		每一层只需要查找一个sstable文件的内容。

		在memory db或者sstable的查找过程中,需要根据指定的序列号拼接一个internalKey,
		查找用户key一致,且seq号不大于指定seq的数据,
		
		\end{enumerate}
	
		\subsubsection{跳表数据结构的实现}

		内存数据库用来维护有序的key-value对,
		其底层是利用跳表实现,绝大多数操作(读/写)的时间复杂度均为O(log n),
		有着与平衡树相媲美的操作效率,但是从实现的角度来说简单许多,
		接下来将介绍一下内存数据库的实现细节。

		\begin{enumerate}
		\item 跳表的实现
		
		跳表(SkipList)是由William Pugh提出的。
		他在论文《Skip lists: a probabilistic alternative to balanced trees》
		中详细地介绍了有关跳表结构、插入删除操作的细节。

		这种数据结构是利用概率均衡技术,加快简化插入、删除操作,
		且保证绝大大多操作均拥有O(log n)的良好效率。
		
		原文的一段话道出了跳表在数据结构中运用离散数学知识的精髓:数据结构是离散的,计算机的本质是离散的。

		\begin{figure}[H]
			\centering
			\includegraphics[width=0.95\textwidth]{images/skiplist_effect}
			\caption{跳表的影响}
			\label{skiplist_effect}
		\end{figure}

		平衡树(以红黑树为代表)是一种非常复杂的数据结构,
		为了维持树结构的平衡,获取稳定的查询效率,平衡树每次插入可能会涉及到较为复杂的节点旋转等操作。
		作者设计跳表的目的就是借助概率平衡,来构建一个快速且简单的数据结构,取代平衡树。

		\begin{figure}[H]
			\centering
			\includegraphics[width=0.95\textwidth]{images/skiplist_intro}
			\caption{一个跳表的图}
			\label{skiplist_intro}
		\end{figure}

		作者从链表讲起,一步步引出了跳表这种结构的由来。

		图a中,所有元素按序排列,被存储在一个链表中,则一次查询之多需要比较N个链表节点;

		图b中,每隔2个链表节点,新增一个额外的指针,该指针指向间距为2的下一个节点,如此以来,借助这些额外的指针,一次查询至多只需要⌈n/2⌉ + 1次比较;

		图c中,在图b的基础上,每隔4个链表节点,新增一个额外的指针,指向间距为4的下一个节点,一次查询至多需要⌈n/4⌉ + 2次比较;

		作者推论,若每隔2个节点,新增一个辅助指针,最终一次节点的查询效率为O(log n)。但是这样不断地新增指针,使得一次插入、删除操作将会变得非常复杂。

		一个拥有k个指针的结点称为一个k层结点(level k node)。按照上面的逻辑,50\% 的结点为1层节点,25\% 的结点为2层节点,12.5\%。
		若保证每层节点的分布如上述概率所示,则仍然能够相同的查询效率。图e便是一个示例。

		维护这些辅助指针将会带来较大的复杂度,因此作者将每一层中,
		每个节点的辅助指针指向该层中下一个节点。
		故在插入删除操作时,只需跟操作链表一样,修改相关的前后两个节点的内容即可完成,
		作者将这种数据结构称为跳表。


		\item 跳表的结构
		
		\begin{figure}[H]
			\centering
			\includegraphics[width=0.95\textwidth]{images/skiplist_arch}
			\caption{跳表的结构}
			\label{skiplist_arch}
		\end{figure}

		跳跃列表是按层建造的。
		底层是一个普通的有序链表。
		每个更高层都充当下面链表的"快速通道",
		这里在层 i 中的元素按某个固定的概率 p (通常为0.5或0.25)出现在层 i+1 中。
		平均起来,每个元素都在 1/(1-p) 个列表中出现,
		而最高层的元素(通常是在跳跃列表前端的一个特殊的头元素)在 O(log1/p n) 个列表中出现。

		\item 跳表的查找
		
		\begin{figure}[H]
			\centering
			\includegraphics[width=0.95\textwidth]{images/skiplist_search}
			\caption{跳表的查找图}
			\label{skiplist_search}
		\end{figure}
		
		在介绍插入和删除操作之前,我们首先介绍查找操作,该操作是上述两个操作的基础。

例如图中,需要查找一个值为17的链表节点,查找的过程为:

首先根据跳表的高度选取最高层的头节点;

若跳表中的节点内容小于查找节点的内容,则取该层的下一个节点继续比较;

若跳表中的节点内容等于查找节点的内容,则直接返回;

若跳表中的节点内容大于查找节点的内容,且层高不为0,则降低层高,
且从前一个节点开始,重新查找低一层中的节点信息;若层高为0,则返回当前节点,
该节点的key大于所查找节点的key。

综合来说,就是利用稀疏的高层节点,快速定位到所需要查找节点的大致位置,
再利用密集的底层节点,具体比较节点的内容。
		

		\item 跳表的插入
		
		跳表的插入以查找为基础实现

		\begin{figure}[H]
			\centering
			\includegraphics[width=0.95\textwidth]{images/skiplist_insert}
			\caption{跳表的插入图}
			\label{skiplist_insert}
		\end{figure}

		在查找的过程中,不断记录每一层的前任节点,如图中红色圆圈所表示的;
		为新插入的节点随机产生层高(随机产生层高的算法较为简单,依赖最高层数和概率值p,可见下文中的代码实现);
		在合适的位置插入新节点(例如图中节点12与节点19之间),并依据查找时记录的前任节点信息,
		在每一层中,以链表插入的方式,将该节点插入到每一层的链接中。


		链表插入指:将当前节点的Next值置为前任节点的Next值,将前任节点的Next值替换为当前节点。

		\begin{lstlisting}[caption=skiplistRandHeight , label=code_radds_storage_skiplist_randHeight]
func (p *DB) randHeight() (h int) {
	const branching = 4
	h = 1
	for h < tMaxHeight && p.rnd.Int()%branching == 0 {
		h++
	}
	return
}	
		\end{lstlisting}

		\item 跳表的删除

		跳表的删除操作较为简单,依赖查找过程找到该节点在整个跳表中的位置后,以链表删除的方式,
		在每一层中,删除该节点的信息。

		链表删除指:将前任节点的Next值替换为当前节点的Next值,并将当前节点所占的资源释放。

		\item 跳表的迭代
		
		(1)向后遍历

		若迭代器刚被创建,则根据用户指定的查找范围[Start, Limit)找到一个符合条件的跳表节点;
		
		若迭代器处于中部,则取出上一次访问的跳表节点的后继节点,
		作为本次访问的跳表节点(后继节点为最底层的后继节点);
		
		利用跳表节点信息(keyvalue数据偏移量,key,value值长度等),获取keyvalue数据;
		
		(2)向前遍历

		若迭代器刚被创建,则根据用户指定的查找范围[Start, Limit)在跳表中找到最后一个符合条件的跳表节点;
		
		若迭代器处于中部,则利用上一次访问的节点的key值,查找比该key值更小的跳表节点;
		
		利用跳表节点信息(keyvalue数据偏移量,key,value值长度等),获取keyvalue数据;

	\end{enumerate}

		\subsubsection{内存数据库的实现}

		在介绍完跳表这种数据结构的组织原理以后,我们介绍存储系统如何利用跳表来构建一个高效的内存数据库。

		\begin{enumerate}
			\item 键值编码
			
			在介绍内存数据库之前,首先介绍一下内存数据库的键值编码规则。
			由于内存数据库本质是一个kv集合,且所有的数据项都是依据key值排序的,因此键值的编码规则尤为关键。

			内存数据库中,key称为internalKey,其由三部分组成:

			用户定义的key:这个key值也就是原生的key值;
			
			序列号:存储系统中,每一次写操作都有一个sequence number,标志着写入操作的先后顺序。
			由于在存储系统中可能会有多条相同key的数据项同时存储在数据库中,
			因此需要有一个序列号来标识这些数据项的新旧情况。序列号最大的数据项为最新值;
			
			类型:标志本条数据项的类型,为更新还是删除;

			\begin{figure}[H]
				\centering
				\includegraphics[width=0.95\textwidth]{images/internalkey}
				\caption{内存数据库内部键}
				\label{internalkey}
			\end{figure}

			\item 键值比较
			
			内存数据库中所有的数据项都是按照键值比较规则进行排序的。
			这个比较规则可以由用户自己定制,也可以使用系统默认的。在这里介绍一下系统默认的比较规则。

			默认的比较规则:

			首先按照字典序比较用户定义的key(ukey),若用户定义key值大,整个internalKey就大;
			
			若用户定义的key相同,则序列号大的internalKey值就小;
			
			通过这样的比较规则,则所有的数据项首先按照用户key进行升序排列;
			当用户key一致时,按照序列号进行降序排列,这样可以保证首先读到序列号大的数据项。


			\item 数据组织
			
			\begin{lstlisting}{caption=storage_db, label=code_radds_storage_db}
type DB struct {
	cmp comparer.BasicComparer
	rnd *rand.Rand
	mu     sync.RWMutex
	kvData []byte
	// Node data:
	// [0]         : KV offset
	// [1]         : Key length
	// [2]         : Value length
	// [3]         : Height
	// [3..height] : Next nodes
	nodeData  []int
	prevNode  [tMaxHeight]int
	maxHeight int
	n         int
	kvSize    int
}
		\end{lstlisting}
				
			
			其中kvData用来存储每一条数据项的key-value数据,nodeData用来存储每个跳表节点的链接信息。

			nodeData中,每个跳表节点占用一段连续的存储空间,每一个字节分别用来存储特定的跳表节点信息。

			第一个字节用来存储本节点key-value数据在kvData中对应的偏移量;
			
			第二个字节用来存储本节点key值长度;
			
			第三个字节用来存储本节点value值长度;
			
			第四个字节用来存储本节点的层高;
			
			第五个字节开始,用来存储每一层对应的下一个节点的索引值;

		\item 基础操作
		
		Put、Get、Delete、Iterator等操作均依赖于底层的跳表的基本操作实现,不再赘述。
		\end{enumerate}

		\subsubsection{持久化数据存储的实现}

			\begin{enumerate}
				\item sstable概述
				
				如我们之前提到的,系统是的LSM树(Log Structured-Merge Tree)实现,
				即一次写入过程并不是直接将数据持久化到磁盘文件中,而是将写操作首先写入日志文件中,
				其次将写操作应用在memtable上。

				当其达到checkpoint点(memtable中的数据量超过了预设的阈值),
				会将当前memtable冻结成一个不可更改的内存数据库
				(immutable memory db),并且创建一个新的memtable供系统继续使用。

				immutable memory db会在后台进行一次minor compaction,
				即将内存数据库中的数据持久化到磁盘文件中。

				LSM树设计Minor Compaction的目的是为了:

				有效地降低内存的使用率;
				
				避免日志文件过大,系统恢复时间过长;
				
				当memory db的数据被持久化到文件中时,
				存储系统将以一定规则进行文件组织,这种文件格式成为sstable。
				在本文中将详细地介绍sstable的文件格式以及相关读写操作。

				\item sstable文件格式
				

				\begin{enumerate}
					\item 物理结构

					为了提高整体的读写效率,一个sstable文件按照固定大小进行块划分,默认每个块的大小为4KiB。
					每个Block中,除了存储数据以外,还会存储两个额外的辅助字段:
	
					压缩类型;CRC校验码
	
					压缩类型说明了Block中存储的数据是否进行了数据压缩,
					若是,采用了哪种算法进行压缩。存储系统中默认采用Snappy算法进行压缩。
	
					CRC校验码是循环冗余校验校验码,校验范围包括数据以及压缩类型。
					
					\begin{figure}[H]
						\centering
						\includegraphics[width=0.95\textwidth]{images/sstable_physic.jpeg}
						\caption{sstable物理结构}
						\label{sstable_physic}
					\end{figure}
					
					\item 逻辑结构
	
					在逻辑上,根据功能不同,存储系统在逻辑上又将sstable分为:
	
	data block: 用来存储key value数据对;
	
	filter block: 用来存储一些过滤器相关的数据(布隆过滤器),但是若用户不指定存储系统使用过滤器,存储系统在该block中不会存储任何内容;
	
	meta Index block: 用来存储filter block的索引信息(索引信息指在该sstable文件中的偏移量以及数据长度);
	
	index block:index block中用来存储每个data block的索引信息;
	
	footer: 用来存储meta index block及index block的索引信息;
	
	\begin{figure}[H]
		\centering
		\includegraphics[width=0.95\textwidth]{images/sstable_logic.jpeg}
		\caption{sstable逻辑结构}
		\label{sstable_logic}
	\end{figure}
	
				每个区块都会有自己的压缩信息以及CRC校验码信息。
	
					\item datablock结构
	
					data block中存储的数据是存储系统中的keyvalue键值对。
					其中一个data block中的数据部分(不包括压缩类型、CRC校验码)按逻辑又以下图进行划分:
					
					\begin{figure}[H]
						\centering
						\includegraphics[width=0.95\textwidth]{images/datablock.jpeg}
						\caption{sstable data block}
						\label{sstable_data_block}
					\end{figure}
					
					第一部分用来存储keyvalue数据。由于sstable中所有的keyvalue对都是严格按序存储的,
					为了节省存储空间,存储系统并不会为每一对keyvalue对都存储完整的key值,而是存储与上一个key非共享的部分,
					避免了key重复内容的存储。
	
					每间隔若干个keyvalue对,将为该条记录重新存储一个完整的key。
					重复该过程(默认间隔值为16),
					每个重新存储完整key的点称之为Restart point。
	
					存储系统设计Restart point的目的是在读取sstable内容时,
					加速查找的过程。
	
					由于每个Restart point存储的都是完整的key值,
					因此在sstable中进行数据查找时,
					可以首先利用restart point点的数据进行键值比较,
					以便于快速定位目标数据所在的区域;
	
					当确定目标数据所在区域时,
					再依次对区间内所有数据项逐项比较key值,进行细粒度地查找;
					该思想有点类似于跳表中利用高层数据迅速定位,
					底层数据详细查找的理念,降低查找的复杂度。
	
					
					每个数据项格式如下图所示:
	
					\begin{figure}[H]
						\centering
						\includegraphics[width=0.95\textwidth]{images/entry_format.jpeg}
						\caption{sstable entry format}
						\label{sstable_entry_format}
					\end{figure}
	
					一个entry分为5部分内容:
	
	与前一条记录key共享部分的长度;
	
	与前一条记录key不共享部分的长度;
	
	value长度;
	
	与前一条记录key非共享的内容;
	
	value内容;
	
	\begin{figure}[H]
		\centering
		\includegraphics[width=0.95\textwidth]{images/datablock_example_1.jpeg}
		\caption{sstable data block 示例图1}
		\label{sstable_datablock_example_1}
	\end{figure}
	
	
	三组entry按上图的格式进行存储。
	值得注意的是restart\_interval为2,
	因此每隔两个entry都会有一条数据作为restart point点的数据项,
	存储完整key值。因此entry3存储了完整的key。
	
	此外,第一个restart point为0(偏移量),第二个restart point为16,
	restart point共有两个,
	因此一个datablock数据段的末尾添加了下图所示的数据:
	
	\begin{figure}[H]
		\centering
		\includegraphics[width=0.95\textwidth]{images/datablock_example_2.jpeg}
		\caption{sstable data block 示例图2}
		\label{sstable_datablock_example_2}
	\end{figure}
	
	尾部数据记录了每一个restart point的值,以及所有restart point的个数。
	
	\item filter block结构
	
	为了加快sstable中数据查询的效率,在直接查询datablock中的内容之前,
	存储系统首先根据filter block中的过滤数据判断指定的datablock中是否有需要查询的数据,
	若判断不存在,则无需对这个datablock进行数据查找。
	
	filter block存储的是data block数据的一些过滤信息。
	这些过滤数据一般指代布隆过滤器的数据,用于加快查询的速度,
	
	\begin{figure}[H]
		\centering
		\includegraphics[width=0.95\textwidth]{images/filterblock_format.jpeg}
		\caption{sstable filterblock format}
		\label{sstable_filterblock_format}
	\end{figure}

	filter block存储的数据主要可以分为两部分:(1)过滤数据(2)索引数据。
	
	其中索引数据中,filter i offset表示第i个filter data在整个filter block中的起始偏移量,filter offset's offset表示filter block的索引数据在filter block中的偏移量。
	
	在读取filter block中的内容时,可以首先读出filter offset's offset的值,然后依次读取filter i offset,根据这些offset分别读出filter data。
	
	Base Lg默认值为11,表示每2KB的数据,创建一个新的过滤器来存放过滤数据。
	
	一个sstable只有一个filter block,其内存储了所有block的filter数据。
	具体来说,filter\_data\_k 包含了所有起始位置处于 [base*k, base*(k+1)]
	范围内的block的key的集合的filter数据,
	按数据大小而非block切分主要是为了尽量均匀,
	以应对存在一些block的key很多,另一些block的key很少的情况。
	
	索引和BloomFilter等元数据可随文件一起创建和销毁,即直接存在文件里,不用加载时动态计算,不用维护更新
				
	\item meta index block结构
	
	meta index block用来存储filter block在整个sstable中的索引信息。

meta index block只存储一条记录:

该记录的key为:"filter."与过滤器名字组成的常量字符串

该记录的value为:filter block在sstable中的索引信息序列化后的内容,
索引信息包括:(1)在sstable中的偏移量(2)数据长度。

	\item index block 结构
	
	与meta index block类似,index block用来存储所有data block的相关索引信息。

indexblock包含若干条记录,每一条记录代表一个data block的索引信息。

一条索引包括以下内容:

data block i 中最大的key值;
该data block起始地址在sstable中的偏移量;
该data block的大小;
\begin{figure}[H]
	\centering
	\includegraphics[width=0.95\textwidth]{images/indexblock_format.jpeg}
	\caption{sstable indexblock format}
	\label{sstable_indexblock_format}
\end{figure}

其中,data block i最大的key值还是index block中该条记录的key值。

如此设计的目的是,依次比较index block中记录信息的key值即可实现快速定位目标数据在哪个data block中。

\item footer结构
footer大小固定,为48字节,用来存储meta index block与index block在sstable中的索引信息,
另外尾部还会存储一个magic word,内容为:"http://code.google.com/p/leveldb/"
字符串sha1哈希的前8个字节。
\begin{figure}[H]
	\centering
	\includegraphics[width=0.95\textwidth]{images/footer_format.jpeg}
	\caption{sstable footer format}
	\label{sstable_footer_format}
\end{figure}


\end{enumerate}
				

				\item sstable读写操作
				
				在介绍完sstable文件具体的组织方式之后,我们再来介绍一下相关的读写操作。
				为了便于理解,将首先介绍写操作。

				\begin{enumerate}
					\item 写操作
					

					sstable的写操作通常发生在:

memory db将内容持久化到磁盘文件中时,会创建一个sstable进行写入;
存储系统后台进行文件compaction时,会将若干个sstable文件的内容重新组织,输出到若干个新的sstable文件中;
对sstable进行写操作的数据结构为tWriter,具体定义如下:

\begin{lstlisting}[caption=tWriter , label=code_radds_storage_tWriter]
// tWriter wraps the table writer. It keep track of file descriptor
// and added key range.
type tWriter struct {
	t *tOps
	
	fd storage.FileDesc // 文件描述符
	w  storage.Writer   // 文件系统writer
	tw *table.Writer
	
	first, last []byte
}
\end{lstlisting}

主要包括了一个sstable的文件描述符,底层文件系统的writer,
该sstable中所有数据项最大最小的key值以及一个内嵌的tableWriter。

一次sstable的写入为一次不断利用迭代器读取需要写入的数据,
并不断调用tableWriter的Append函数,
直至所有有效数据读取完毕,为该sstable文件附上元数据的过程。

该迭代器可以是一个内存数据库的迭代器,写入情景对应着上述的第一种情况;

该迭代器也可以是一个sstable文件的迭代器,写入情景对应着上述的第二种情况;

sstable的元数据包括:(1)文件编码(2)大小(3)最大key值(4)最小key值

故,理解tableWriter的Append函数是理解整个写入过程的关键。


			\begin{enumerate}
				\item tableWriter
				
				在介绍append函数之前,首先介绍一下tableWriter这个数据结构。主要的定义如下:
\begin{lstlisting}[caption=Writer , label=code_radds_storage_Writer]
// Writer is a table writer.
type Writer struct {
	writer io.Writer
	// Options
	blockSize   int // 默认是4KiB

	dataBlock   blockWriter // data块Writer
	indexBlock  blockWriter // indexBlock块Writer
	filterBlock filterWriter // filter块Writer
	pendingBH   blockHandle
	offset      uint64
	nEntries    int // key-value键值对个数
}				
\end{lstlisting}

其中blockWriter与filterWriter表示底层的两种不同的writer,blockWriter负责写入data数据的写入,而filterWriter负责写入过滤数据。

pendingBH记录了上一个dataBlock的索引信息,当下一个dataBlock的数据开始写入时,将该索引信息写入indexBlock中。
				\item Append 
				
				一次append函数的主要逻辑如下:

若本次写入为新dataBlock的第一次写入,则将上一个dataBlock的索引信息写入;
将keyvalue数据写入datablock;
将过滤信息写入filterBlock;
若datablock中的数据超过预定上限,则标志着本次datablock写入结束,将内容刷新到磁盘文件中;

\begin{lstlisting}[caption=Append , label=code_radds_storage_Append]
func (w *Writer) Append(key, value []byte) error {
	w.flushPendingBH(key)
	// Append key/value pair to the data block.
	w.dataBlock.append(key, value)
	// Add key to the filter block.
	w.filterBlock.add(key)
	
	// Finish the data block if block size target reached.
	if w.dataBlock.bytesLen() >= w.blockSize {
		if err := w.finishBlock(); err != nil {
			w.err = err
			return w.err
		}
	}
	w.nEntries++
	return nil
}
\end{lstlisting}


\item dataBlock.append

该函数将编码后的kv数据写入到dataBlock对应的buffer中,编码的格式如上文中提到的数据项的格式。此外,在写入的过程中,若该数据项为restart点,则会添加相应的restart point信息。

\item filterBlock.append

该函数将kv数据项的key值加入到过滤信息中,具体可见《存储系统源码解析 - 布隆过滤器》

\item finishBlock

若一个datablock中的数据超过了固定上限,则需要将相关数据写入到磁盘文件中。

在写入时,需要做以下工作:

封装dataBlock,记录restart point的个数;
若dataBlock的数据需要进行压缩(例如snappy压缩算法),则对dataBlock中的数据进行压缩;
计算checksum;
封装dataBlock索引信息(offset,length);
将datablock的buffer中的数据写入磁盘文件;
利用这段时间里维护的过滤信息生成过滤数据,放入filterBlock对用的buffer中;

\item Close

当迭代器取出所有数据并完成写入后,调用tableWriter的Close函数完成最后的收尾工作:

若buffer中仍有未写入的数据,封装成一个datablock写入;
将filterBlock的内容写入磁盘文件;
将filterBlock的索引信息写入metaIndexBlock中,写入到磁盘文件;
写入indexBlock的数据;
写入footer数据;
至此为止,所有的数据已经被写入到一个sstable中了,由于一个sstable是作为一个memory db或者Compaction的结果原子性落地的,因此在sstable写入完成之后,将进行更为复杂的存储系统的版本更新,将在接下来的文章中继续介绍。
			\end{enumerate}


				\item 读操作
				
				读操作作为写操作的逆过程,充分理解了写操作,将会帮助理解读操作。

下图为在一个sstable中查找某个数据项的流程图:

\begin{figure}[H]
	\centering
	\includegraphics[width=0.95\textwidth]{images/sstable_read_procedure.jpeg}
	\caption{sstable read procedure}
	\label{sstable_read_procedure}
\end{figure}

大致流程为:

首先判断“文件句柄”cache中是否有指定sstable文件的文件句柄,若存在,则直接使用cache中的句柄;

否则打开该sstable文件,按规则读取该文件的元数据,将新打开的句柄存储至cache中;

利用sstable中的index block进行快速的数据项位置定位,得到该数据项有可能存在的两个data block;

利用index block中的索引信息,首先打开第一个可能的data block;

利用filter block中的过滤信息,判断指定的数据项是否存在于该data block中,若存在,则创建一个迭代器对data block中的数据进行迭代遍历,寻找数据项;若不存在,则结束该data block的查找;

若在第一个data block中找到了目标数据,则返回结果;若未查找成功,则打开第二个data block,重复步骤4;

若在第二个data block中找到了目标数据,则返回结果;若未查找成功,则返回Not Found错误信息;

		\begin{enumerate}
			\item 缓存 
			

			在存储系统中,使用cache来缓存两类数据:

sstable文件句柄及其元数据;
data block中的数据;
因此在打开文件之前,首先判断能够在cache中命中sstable的文件句柄,避免重复读取的开销。

			\item 元数据读取 
			

			\begin{figure}[H]
				\centering
				\includegraphics[width=0.95\textwidth]{images/sstable_metadata.jpeg}
				\caption{sstable metadata}
				\label{sstable_metadata}
			\end{figure}
			由于sstable复杂的文件组织格式,因此在打开文件后,需要读取必要的元数据,才能访问sstable中的数据。

元数据读取的过程可以分为以下几个步骤:

读取文件的最后48字节的利用,即Footer数据;

读取Footer数据中维护的(1) Meta Index Block(2) Index Block两个部分的索引信息并记录,以提高整体的查询效率;

利用meta index block的索引信息读取该部分的内容;

遍历meta index block,查看是否存在“有用”的filter block的索引信息,若有,则记录该索引信息;

若没有,则表示当前sstable中不存在任何过滤信息来提高查询效率;

			\item 数据项的快速定位 
			
			sstable中存在多个data block,倘若依次进行“遍历”显然是不可取的。但是由于一个sstable中所有的数据项都是按序排列的,因此可以利用有序性已经index block中维护的索引信息快速定位目标数据项可能存在的data block。

一个index block的文件结构示意图如下:

\begin{figure}[H]
	\centering
	\includegraphics[width=0.95\textwidth]{images/indexblock.jpeg}
	\caption{sstable index block}
	\label{sstable_index_block}
\end{figure}

index block是由一系列的键值对组成,每一个键值对表示一个data block的索引信息。

键值对的key为该data block中数据项key的最大值,value为该data block的索引信息(offset, length)。

因此若需要查找目标数据项,仅仅需要依次比较index block中的这些索引信息,
倘若目标数据项的key大于某个data block中最大的key值,则该data block中必然不存在目标数据项。
故通过这个步骤的优化,可以直接确定目标数据项落在哪个data block的范围区间内。

值得注意的是,与data block一样,index block中的索引信息同样也进行了key值截取,即第二个索引信息的key并不是存储完整的key,而是存储与前一个索引信息的key不共享的部分,区别在于data block中这种范围的划分粒度为16,而index block中为2 。

也就是说,index block连续两条索引信息会被作为一个最小的“比较单元“,在查找的过程中,若第一个索引信息的key小于目标数据项的key,则紧接着会比较第三条索引信息的key。

这就导致最终目标数据项的范围区间为某”两个“data block。

\begin{figure}[H]
	\centering
	\includegraphics[width=0.95\textwidth]{images/index_block_find.jpeg}
	\caption{sstable index block find}
	\label{sstable_index_block_find}
\end{figure}

	\item 过滤data
	
	若sstable存有每一个data block的过滤数据,则可以利用这些过滤数据对data block中的内容进行判断,“确定”目标数据是否存在于data block中。

过滤的原理为:

若过滤数据显示目标数据不存在于data block中,则目标数据一定不存在于data block中;
若过滤数据显示目标数据存在于data block中,则目标数据可能存在于data block中;
具体的原理可能参见《布隆过滤器》。

因此利用过滤数据可以过滤掉部分data block,避免发生无谓的查找。

	\item 查找datablock
	
	\begin{figure}[H]
		\centering
		\includegraphics[width=0.95\textwidth]{images/datablock_format.jpeg}
		\caption{sstable datablock format}
		\label{sstable_datablock_format}
	\end{figure}
	在data block中查找目标数据项是一个简单的迭代遍历过程。虽然data block中所有数据项都是按序排序的,但是作者并没有采用“二分查找”来提高查找的效率,而是使用了更大的查找单元进行快速定位。

与index block的查找类似,data block中,以16条记录为一个查找单元,若entry 1的key小于目标数据项的key,则下一条比较的是entry 17。

因此查找的过程中,利用更大的查找单元快速定位目标数据项可能存在于哪个区间内,之后依次比较判断其是否存在与data block中。

可以看到,sstable很多文件格式设计(例如restart point, index block,filter block,max key)在查找的过程中,都极大地提升了整体的查找效率。



		\end{enumerate}
			
				\end{enumerate}
				

				\item sstable文件特点
				
				\begin{enumerate}
					\item 只读性

sstable文件为compaction的结果原子性的产生,在其余时间是只读的。

\item 完整性

一个sstable文件,其辅助数据:

索引数据和过滤数据都直接存储于同一个文件中。
当读取是需要使用这些辅助数据时,无须额外的磁盘读取;
当sstable文件需要删除时,无须额外的数据删除。
简要地说,辅助数据随着文件一起创建和销毁。

\item 并发访问友好性

由于sstable文件具有只读性,因此不存在同一个文件的读写冲突。

存储系统采用引用计数维护每个文件的引用情况,当一个文件的计数值大于0时,对此文件的删除动作会等到该文件被释放时才进行,因此实现了无锁情况下的并发访问。

\item Cache一致性

sstable文件为只读的,因此cache中的数据永远于sstable文件中的数据保持一致。
				\end{enumerate}
			\end{enumerate}


		\subsubsection{缓存系统的实现}

		缓存对于一个数据库读性能的影响十分巨大,倘若存储系统的每一次读取都会发生一次磁盘的IO,那么其整体效率将会非常低下。

		存储系统中使用了一种基于LRUCache的缓存机制,用于缓存:
		
		已打开的sstable文件对象和相关元数据;
		sstable中的dataBlock的内容;
		使得在发生读取热数据时,尽量在cache中命中,避免IO读取。在介绍如何缓存、利用这些数据之前,我们首先介绍一下leveldb使用的cache是如何实现的。
		
		\begin{enumerate}
			\item Cache结构 
			
			存储系统中使用的cache是一种LRUcache,其结构由两部分内容组成:

			Hash table:用来存储数据;
			
			LRU:用来维护数据项的新旧信息;

			\begin{figure}[H]
				\centering
				\includegraphics[width=0.95\textwidth]{images/cache_arch.jpeg}
				\caption{sstable cache arch}
				\label{sstable_cache_arch}
			\end{figure}

			其中Hash table是基于Yujie Liu等人的论文《Dynamic-Sized Nonblocking Hash Table》实现的,用来存储数据。由于hash表一般需要保证插入、删除、查找等操作的时间复杂度为 O(1)。

当hash表的数据量增大时,为了保证这些操作仍然保有较为理想的操作效率,需要对hash表进行resize,即改变hash表中bucket的个数,对所有的数据进行重散列。

基于该文章实现的hash table可以实现resize的过程中不阻塞其他并发的读写请求。

LRU中则根据Least Recently Used原则进行数据新旧信息的维护,当整个cache中存储的数据容量达到上限时,便会根据LRU算法自动删除最旧的数据,使得整个cache的存储容量保持一个常量。


			\item Dynamic-sized NonBlocking Hash table
			
			在hash表进行resize的过程中,保持Lock-Free是一件非常困难的事。

一个hash表通常由若干个bucket组成,每一个bucket中会存储若干条被散列至此的数据项。当hash表进行resize时,需要将“旧”桶中的数据读出,并且重新散列至另外一个“新”桶中。假设这个过程不是一个原子操作,那么会导致此刻其他的读、写请求的结果发生异常,甚至导致数据丢失的情况发生。

因此,liu等人提出了一个新颖的概念:一个bucket的数据是可以冻结的。

这个特点极大地简化了hash表在resize过程中在不同bucket之间转移数据的复杂度。

			\begin{enumerate}
				\item 散列 
				
				\begin{figure}[H]
					\centering
					\includegraphics[width=0.95\textwidth]{images/cache_select.jpeg}
					\caption{sstable cache select}
					\label{sstable_cache_select}
				\end{figure}
				该哈希表的散列与普通的哈希表一致,都是借助散列函数,将用户需要查找、更改的数据散列到某一个哈希桶中,并在哈希桶中进行操作。

由于一个哈希桶的容量是有限的(一般不大于32个数据),因此在哈希桶中进行插入、查找的时间复杂度可以视为是常量的。

				
				\item 扩大
				
				\begin{figure}[H]
					\centering
					\includegraphics[width=0.95\textwidth]{images/cache_expend.jpeg}
					\caption{sstable cache expand}
					\label{sstable_cache_expand}
				\end{figure}
				当cache中维护的数据量太大时,会发生哈希表扩张的情况。以下两种情况是为“cache中维护的数据量过大”:

整个cache中,数据项(node)的个数超过预定的阈值(默认初始状态下哈希桶的个数为16个,每个桶中可存储32个数据项,即总量的阈值为哈希桶个数乘以每个桶的容量上限);
当cache中出现了数据不平衡的情况。当某些桶的数据量超过了32个数据,即被视作数据发生散列不平衡。当这种不平衡累积值超过预定的阈值(128)个时,就需要进行扩张;
一次扩张的过程为:

计算新哈希表的哈希桶个数(扩大一倍);
创建一个空的哈希表,并将旧的哈希表(主要为所有哈希桶构成的数组)转换一个“过渡期”的哈希表,表中的每个哈希桶都被“冻结”;
后台利用“过渡期”哈希表中的“被冻结”的哈希桶信息对新的哈希表进行内容构建;
值得注意的是,在完成新的哈希表构建的整个过程中,哈希表并不是拒绝服务的,所有的读写操作仍然可以进行。

哈希表扩张过程中,最小的封锁粒度为哈希桶级别。

当有新的读写请求发生时,若被散列之后得到的哈希桶仍然未构建完成,则“主动”进行构建,并将构建后的哈希桶填入新的哈希表中。后台进程构建到该桶时,发现已经被构建了,则无需重复构建。

因此如上图所示,哈希表扩张结束,哈希桶的个数增加了一倍,于此同时仍然可以对外提供读写服务,仅仅需要哈希桶级别的封锁粒度就可以保证所有操作的一致性跟原子性。

构建哈希桶

当哈希表扩张时,构建一个新的哈希桶其实就是将一个旧哈希桶中的数据拆分成两个新的哈希桶。

拆分的规则很简单。由于一次散列的过程为:

利用散列函数对数据项的key值进行计算;
将第一步得到的结果取哈希桶个数的余,得到哈希桶的ID;
因此拆分时仅需要将数据项key的散列值对新的哈希桶个数取余即可。
				
				\item 缩小 
				
				当哈希表中数据项的个数少于哈希桶的个数时,需要进行收缩。
				收缩时,哈希桶的个数变为原先的一半,2个旧哈希桶的内容被合并成一个新的哈希桶,
				过程与扩张类似,在这里不展开详述。

			\end{enumerate}
		
		\item LRU 
		
		除了利用哈希表来存储数据以外,存储系统还利用LRU来管理数据。

存储系统中,LRU利用一个双向循环链表来实现。每一个链表项称之为LRUNode。

\begin{lstlisting}[caption=lruNode , label=code_radds_storage_lruNode]
type lruNode struct {
	n   *Node // customized node
	h   *Handle
	ban bool

	next, prev *lruNode
}
\end{lstlisting}

一个LRUNode除了维护一些链表中前后节点信息以外,还存储了一个哈希表中数据项的指针,通过该指针,当某个节点由于LRU策略被驱逐时,从哈希表中“安全的”删除数据内容。

LRU提供了以下几个接口:

Promote
若一个hash表中的节点是第一次被创建,则为该节点创建一个LRUNode,并将LRUNode置于链表的头部,表示为最新的数据;

若一个hash表中的节点之前就有相关的LRUNode存在与链表中,将该LRUNode移至链表头部;

若因为新增加一个LRU数据,导致超出了容量上限,就需要根据策略清除部分节点。

Ban
将hash表节点对应的LRUNode从链表中删除,并“尝试”从哈希表中删除数据。

由于该哈希表节点的数据可能被其他线程正在使用,因此需要查看该数据的引用计数,只有当引用计数为0时,才可以真正地从哈希表中进行删除。

		\item 缓存数据 
		
		存储系统利用上述的cache结构来缓存数据。其中:

cache:来缓存已经被打开的sstable文件句柄以及元数据(默认上限为500个);

bcache:来缓存被读过的sstable中dataBlock的数据(默认上限为8MB);

当一个sstable文件需要被打开时,首先从cache中寻找是否已经存在相关的文件句柄,若存在则无需重复打开;

若不存在,则从打开相关文件,并将(1)indexBlock数据,(2)metaIndexBlock数据等相关元数据进行预读。

		\end{enumerate}
		

		\subsubsection{布隆过滤器的实现}


		Bloom Filter是一种空间效率很高的随机数据结构,
		它利用位数组很简洁地表示一个集合,并能判断一个元素是否属于这个集合。
		Bloom Filter的这种高效是有一定代价的:
		在判断一个元素是否属于某个集合时,
		有可能会把不属于这个集合的元素误认为属于这个集合(false positive)。
		因此,Bloom Filter不适合那些“零错误”的应用场合。
		而在能容忍低错误率的应用场合下,
		Bloom Filter通过极少的错误换取了存储空间的极大节省。

leveldb中利用布隆过滤器判断指定的key值是否存在于sstable中,
若过滤器表示不存在,则该key一定不存在,由此加快了查找的效率。

		\begin{enumerate}
			\item 结构 
			
			bloom过滤器底层是一个位数组,初始时每一位都是0 

			\begin{figure}[H]
				\centering
				\includegraphics[width=0.95\textwidth]{images/bloom1}
				\caption{sstable bloom1}
				\label{sstable_bloom1}
			\end{figure}

当插入值x后,分别利用k个哈希函数(图中为3)利用x的值进行散列,并将散列得到的值与bloom过滤器的容量进行取余,将取余结果所代表的那一位值置为1。

\begin{figure}[H]
	\centering
	\includegraphics[width=0.95\textwidth]{images/bloom2}
	\caption{sstable bloom2}
	\label{sstable_bloom2}
\end{figure}

一次查找过程与一次插入过程类似,同样利用k个哈希函数对所需要查找的值进行散列,只有散列得到的每一个位的值均为1,才表示该值“有可能”真正存在;反之若有任意一位的值为0,则表示该值一定不存在。例如y1一定不存在;而y2可能存在。

\begin{figure}[H]
	\centering
	\includegraphics[width=0.95\textwidth]{images/bloom3}
	\caption{sstable bloom3}
	\label{sstable_bloom3}
\end{figure}

			\item 数学结论
			
			http://blog.csdn.net/jiaomeng/article/details/1495500该文中从数学的角度阐述了布隆过滤器的原理,以及一系列的数学结论。

首先,与布隆过滤器准确率有关的参数有:

哈希函数的个数k;
布隆过滤器位数组的容量m;
布隆过滤器插入的数据数量n;
主要的数学结论有:

为了获得最优的准确率,当k = ln2 * (m/n)时,布隆过滤器获得最优的准确性;
在哈希函数的个数取到最优时,要让错误率不超过є,m至少需要取到最小值的1.44倍;

			\item 代码实现
			
			leveldb中的布隆过滤器实现较为简单,以goleveldb为例,
			有关的代码在filter/bloom.go中。

定义如下,bloom过滤器只是一个int数字。

\begin{lstlisting}[caption=tFile , label=code_radds_storage_typedef_bloomfilter]
	type bloomFilter int
\end{lstlisting}

创建一个布隆过滤器时,只需要指定为每个key分配的位数即可,如结论2所示,只要该值(m/n)大于1.44即可,一般可以取10。

\begin{lstlisting}[caption=NewBloomFilter , label=code_radds_storage_newbloomfilter]
func NewBloomFilter(bitsPerKey int) Filter {
	return bloomFilter(bitsPerKey)
}
\end{lstlisting}

创建一个generator, 这一步中需要指定哈希函数的个数k,可以看到k = f * ln2,而f = m/n,即数学结论1。

返回的generator中可以添加新的key信息,调用generate函数时,将所有的key构建成一个位数组写在指定的位置。

\begin{lstlisting}[caption=NewGenerator , label=code_radds_storage_NewGenerator]
func (f bloomFilter) NewGenerator() FilterGenerator {
	// Round down to reduce probing cost a little bit.
	k := uint8(f * 69 / 100) // 0.69 =~ ln(2)
	if k < 1 {
		k = 1
	} else if k > 30 {
		k = 30
	}
	return &bloomFilterGenerator{
		n: int(f),
		k: k,
	}
}
\end{lstlisting}

generator主要有两个函数:

Add

Generate

Add函数中,只是简单地将key的哈希散列值存储在一个整型数组中

\begin{lstlisting}[caption=Add , label=code_radds_storage_Add]
func (g *bloomFilterGenerator) Add(key []byte) {
	// Use double-hashing to generate a sequence of hash values.
	// See analysis in [Kirsch,Mitzenmacher 2006].
	g.keyHashes = append(g.keyHashes, bloomHash(key))
}
\end{lstlisting}


Generate函数中,将之前一段时间内所有添加的key信息用来构建一个位数组,该位数组中包含了所有key的存在信息。

位数组的大小为用户指定的每个key所分配的位数 乘以 key的个数。

位数组的最末尾用来存储k的大小。

\begin{lstlisting}[caption=Generate , label=code_radds_storage_Generate]
func (g *bloomFilterGenerator) Generate(b Buffer) {
	// Compute bloom filter size (in both bits and bytes)
	// len(g.keyHashes) 可以理解为n, g.n可以理解为m/n
	// nBits可以理解为m
	nBits := uint32(len(g.keyHashes) * g.n)
	// For small n, we can see a very high false positive rate.  Fix it
	// by enforcing a minimum bloom filter length.
	if nBits < 64 {
		nBits = 64
	}
	nBytes := (nBits + 7) / 8
	nBits = nBytes * 8

	dest := b.Alloc(int(nBytes) + 1)
	dest[nBytes] = g.k

	for _, kh := range g.keyHashes {
		// Double Hashing
		delta := (kh >> 17) | (kh << 15) // Rotate right 17 bits
		for j := uint8(0); j < g.k; j++ {
			bitpos := kh % nBits
			dest[bitpos/8] |= (1 << (bitpos % 8))
			kh += delta
		}
	}

	g.keyHashes = g.keyHashes[:0]
}
\end{lstlisting}


Contain函数用来判断指定的key是否存在。

\begin{lstlisting}[caption=tFile , label=code_radds_storage_tfile]
func (f bloomFilter) Contains(filter, key []byte) bool {
	nBytes := len(filter) - 1
	if nBytes < 1 {
	    return false
	}
	nBits := uint32(nBytes * 8)
	// Use the encoded k so that we can read filters generated by
	// bloom filters created using different parameters.
	k := filter[nBytes]
	if k > 30 {
	    // Reserved for potentially new encodings for short bloom filters.
	    // Consider it a match.
	    return true
	}
	kh := bloomHash(key)
	delta := (kh >> 17) | (kh << 15) // Rotate right 17 bits
	for j := uint8(0); j < k; j++ {
	    bitpos := kh % nBits
	    if (uint32(filter[bitpos/8]) & (1 << (bitpos % 8))) == 0 {
	        return false
	    }
	    kh += delta
	}
	return true
}
\end{lstlisting}

		\end{enumerate}

		\subsubsection{数据压缩系统的实现}

		\subsubsection{版本控制的实现}

	
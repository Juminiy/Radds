\section{绪论}
  	\subsection{课题的背景和意义}
	  分布式存储系统是一个重要且不断发展的领域,它涉及到许多学科和技术,如计算机科学、网络安全、数据库管理等。
	  分布式存储系统的研究主要集中在三个方面:分布式存储系统的架构设计、底层协议的研究以及应用场景的拓展。其中,架构设计包括存储系统的硬件架构、软件框架等;底层协议的研究则包括各种存储接口、协议栈等;应用场景的拓展则涵盖了企业级、消费级、个人级等不同类型的存储系统。
	  分布式存储系统的研究具有广泛的应用前景,它可以应用于企业级存储系统、云存储服务和分布式应用等方面。在企业级存储系统中,分布式存储系统可以用于存储大量的数据,提高数据存储的效率和可靠性。在云存储服务中,分布式存储系统可以用于存储云端的数据,提供高效的数据存储和管理服务。在分布式应用中,分布式存储系统可以用于存储和管理大量的数据,提高应用的可靠性和性能。
	  在分布式存储系统的研究中,需要解决的关键问题包括数据一致性、数据持久性、数据备份和恢复等。在解决这些问题的过程中,需要采用一些新的技术和方法,如分布式数据库、分布式文件系统、分布式对象存储等。同时,需要考虑分布式存储系统的安全性和可靠性问题,采用一些新的安全技术和机制,如安全协议、数据加密、访问控制等,以保证分布式存储系统的安全性和可靠性。
	
	
	
	\subsection{分布式存储系统的发展状况}
	
	分布式存储系统的发展历史可以追溯到上世纪90年代,当时出现了一些基于局部存储器的分布式存储系统,如Lustre和Xanadu等。这些系统主要用于文件服务器等领域。随着网络技术的发展,基于网络的分布式存储系统出现了,如Hadoop和HDFS等。这些系统将数据存储在分布式节点上,并通过网络进行数据的访问和管理。
	近年来,随着云计算的发展,分布式存储系统开始广泛应用于云存储服务中。云存储服务将数据存储在云端,并通过互联网提供数据的访问和管理。这些系统通常采用分布式文件系统,如AFS、HDFS和Gluster等。
	随着大数据时代的到来,分布式存储系统的研究和应用也越来越广泛。基于分布式文件系统的分布式存储系统可以存储海量的数据,并支持高效的数据存储和管理。同时,基于块存储的分布式存储系统也得到了广泛的研究和应用,它可以实现数据的高效复制和同步,并支持大规模的数据存储和管理。
	总之,分布式存储系统的发展历史可以分为三个阶段:基于局部存储器的分布式存储系统、基于网络的分布式存储系统和基于块存储的分布式存储系统。随着大数据时代的到来,分布式存储系统的研究和应用也将越来越广泛和深入。
	
  	\subsection{课题研究的主要方法及内容}

 	本课题主要工作是...

  	本课题主要包含以下几个方面内容:
  
  	\begin{enumerate}
		\item xxx 
		\item xxxx
		\item xxxxx
  	\end{enumerate}

  	\subsection{论文组织结构}
  
  	本文主要围绕相关技术选型,需求分析,系统整体设计、详细设计,部署与测试等方面来进行论述,共分为6章,各章内容如下:
	
	第1章...
	
    第2章...
    
    第3章...
    
    第4章... 
       
    第5章...
    
    第6章...
    
    为了更好的理解 ...
    
\clearpage